\chapter{Result and Discussion} % Main chapter title

\label{Chapter2.75} % Change X to a consecutive number; for referencing this chapter else where, use \ref{ChapterX}
%----------------------------------------------------------------------------------------
%	SECTION 1
%----------------------------------------------------------------------------------------

%-----------------------------------
%	SECTION 2
%-----------------------------------
\section{Compilation of data}
To investigate the spectral diffusion behaviour, recorded time-resolved spectra for each of the POI before and after treatment. After the first oxidation, for better characterisation, time-resolved spectra are measured not only with one but multiple excitation polarisation. This leaves large amount of spectra files (almost 32,000 spectra!), besides visualizing the movement of lines into color maps, we need better tools to evaluate the spectral stability, a way to compile dozens, sometimes even hundreds of spectra into one single value that represents the spectral stability of a POI.

Normalized cross-correlation evaluates the similarity between 2 patterns, the closer to 1, the higher the similarity. Here we compare the time-resolved PL spectra from every time tick with the first spectrum, the average value of this series of normalized cross-correlation reflects the general level of spectral stability of the measured point.


\section{Nanodiamond size and spectral stability}

The observation of nanodiamond with SEM [fig.] fits the expectation, that after the centrifuging size selection, batch1 contains smaller nanodiamonds than batch2.  
When comparing the time-resolved spectra from sample1508 and sample1510, which were taken at 4K, with an excitation power of 120$\pm$10uW, it can be observed that the spectra from sample1510 shows more spectral diffusion. The calculation of mean normalized cross-correlation confirmed the observation. As plotted in the histogram \ref{fig:histogram-of-normalized-cross-correlation12}, the distribution of mean normalized cross-correlation nanodiamond from batch1 is much closer to 1 than from nanodiamond batch2.

As discussed in Chapter 1, in nitrogen doped diamond, to match up the chemical potential inside the diamond and on the surface, a depletion layer was formed beneath the surface, where the valence band and conducting band are bended towards up. The width of the depletion layer differs by the concentration of the donor and is around 80nm in moderately doped diamond (N$_{D}$=10$^{-16}/cm^{3}$)[Diederich,1998]. 

Since the donor level of nitrogen is rather deep (1.7eV) below the minimum of conduction band, it is very hard to thermally ionize the donors at low temperature, but with the excitation of 532nm(2.33eV) green laser, it is possible to excite the electrons from the donor level to the conduction band. 

So in our case, there is higher chance for nanodiamonds from batch1, which are of smaller sizes, to obtain SiV$^{-}$s sitting inside the bended band than those ones of larger diameters, like the ones from batch2. For more information I looked up the position of SiV$^{-}$s in the paper [Rogers2013], where SiV$^{-}$s in bulk diamond are reported to have very good long term spectral stability (no spectral position variation in 90min), interestingly, the distance of SiV$^{-}$s from the surface is larger than 2um, which means they sat in the bulk diamond region instead of band bended region. 


\section{Excitation power and spectral stability}

Sample 1508 was excited with decreasing excitation power. The left column of \ref{fig:powerdependenceref11} shows the changes of time-resolved spectra of POI ref11, the most obviously diffusing line is the one between 738nm and 739nm, whose spectral position changed almost 1nm in 300s. As the excitation power decreases, the diffusion become less intense.

The result of cross correlation is can be seen in \ref{powerdependencybeforeafteroxidation}, where each dot represents a POI in the sample. As shown in the figure, the general level of spectral diffusion decreases as the excitation power decreases. 

As I try to explain the spectral behaviour with the band bending layer, the power dependency of spectral diffusion can be explained by the photon induced charge transfer between the conducting band and the surface, which will enhance the band bending, thus with higher excitation power, comes greater spectral diffusion. It is also shown in the figure, that after the oxidation, the response from the spectral diffusion towards the excitation power change is much greater, this gives evidence for the explanation of next observation: temperature change against spectral diffusion, which will be discussed in the next section.  
 
\begin{figure}[h]
\centering
\includegraphics[width=0.7\linewidth]{Figures/pic/powerdependencybeforeafteroxidation}
\caption{Powerdependency of nanodiamond batch2, before and after first oxidation.}
\label{fig:powerdependencybeforeafteroxidation}
\end{figure}

\section{Temperature and spectral stability}
Comparing the mean normalized cross-correlation of sample 1510 from 4.7K and 20K in \ref{fig:histogram-of-normalized-cross-correlation12}, no significant difference between the two temperatures has been found,which fits the expectation. Since the assumption is that, the donors are ionized with the help of laser, the band bending should stay the same as long as the excitation power is not changed, so shall the spectral diffusion.
I consider this result not solid enough as only 3 points were measured for the 4.7K measurement. But it would be interesting to collect more data to justify the prediction.

\paragraph{excitation polarisation}

\begin{figure}[h]
\centering
\includegraphics[width=0.7\linewidth]{"Figures/pic/excitation polarisation"}
\caption{Mean normalized cross-correlation against excitation polarisation. The error bar stand for standard deviation}
\label{fig:excitation-polarisation of untreated nanodiamond batch 2}
\end{figure}

The polarisation of photoluminescence has been measured in [L.J.Lachlan,2014], where the excitation polarisation patterns suggested that the SiV$^{-}$ is aligned along the <111> axis of diamond crystal. A set of dots of none-polarisation dependency(excite from the z axis of the SiV$^{-}$), or symmetric patterns that are rotationally separate from each other by an angle of 60$^{o}$ with population contrast of 60$\%$ are expected to be observed when excite the SiV$^{-}$ along crystal axis <111>. The polarisation pattern generated from other orientations can be considered as shadowing the <111> axis from respective angle.

In the measurement of excitation polarisation on sample 1510, time-resolved PL spectra are taken with different excitation polarisations. We do noticed some POIs behaved differently when excited with different beam polarisation. But no clear polarisation pattern has been acquired. Further cross-correlation calculation \ref{fig:excitation-polarisation of untreated nanodiamond batch 2} showed polarisation dependency like pattern in poi ref12, while the other two pois have no clear pattern.

Further excitation polarisation-resolved spectra were taken on the same sample. In some of the spectra, polarisation-related periodic pattern can be seen on some of the lines, while the general pattern give no polarisation dependency.

\begin{figure}[h]
\centering
\includegraphics[width=1\linewidth]{Figures/pic/polarisationexcitation}
\caption{4 Excitation polarisation-resolved spectra from sample 1510, the }
\label{fig:polarisationexcitation}
\end{figure}
 


\begin{figure}[h]
\centering
\includegraphics[width=1\linewidth]{Figures/pic/powerdependenceref11}
\caption{Left column: time resoved PL spectra of point ref11, sample1508 before the first aerobic oxdiation with different excitation power (A:250uW, B:122uW, C:60uW, D:30uW). Right column:time resoved PL spectra of point ref11, sample1508 after the first aerobic oxdiation with different excitation power (A:250uW, B:120uW, C:60uW, D:25uW).}
\label{fig:powerdependenceref11}
\end{figure}
\begin{figure}[h]
\centering
\includegraphics[width=0.7\linewidth]{"Figures/pic/Histogram of normalized cross-correlation_1_2"}
\caption{Histogram comparing the normalized mean value of cross-correlation between untreated sample 1508 (nanodiamond batch2) at 4.7K and sample 1510(nanodiamond batch1) at 4.7K and 20K. The height of the bars are normalized by the integration over bins. The time resolved spectra of sample 1510 was recorded with 2 different excitation polarisation, they are separated with solid and dashed lines.}
\label{fig:histogram-of-normalized-cross-correlation12}
\end{figure}
\begin{figure}[h]
\centering
\includegraphics[width=0.7\linewidth]{"Figures/pic/Histogram of normalized cross-correlation_1_H"}
\caption{Histogram comparing the normalized mean value of cross-correlation between untreated sample 1510 and hydrogen plasma treated sample 1512 (both spin coated with nanodiamond from the batch1).The height of the bars are normalized by the integration over bins.  }
\label{fig:histogram-of-normalized-cross-correlation1h}
\end{figure}
\begin{figure}[h]
\centering
\includegraphics[width=0.7\linewidth]{"Figures/pic/Histogram of normalized cross-correlation_1_H_2"}
\caption{Histogram comparing the normalized mean value of cross-correlation between untreated sample 1508(nanodiamond batch2) and hydrogen plasma treated sample 1512(nanodiamond batch1).The height of the bars are normalized by the integration over bins.  }
\label{fig:histogram-of-normalized-cross-correlation1h2}
\end{figure}

\begin{figure}[h]
\centering
\includegraphics[width=0.7\linewidth]{"Figures/pic/Histogram of normalized cross-correlation_2_o"}
\caption{Histogram comparing the normalized mean value of cross-correlation between untreated and oxidized sample 1509(nanodiamond batch1) at 4.7K. The height of the bars are normalized by the integration over bins. }
\label{fig:histogram-of-normalized-cross-correlation2o}
\end{figure}


\section{Oxidation}
\section{Hydrogen termination}


